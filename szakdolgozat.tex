\documentclass[a4paper,12pt]{report}

\usepackage[latin2]{inputenc} % vagy latin2 helyett utf8
\usepackage[T1]{fontenc}      % karakterkódolás
%\usepackage[magyar]{babel}    % magyar beállítások
\frenchspacing                % helyközök
%\usepackage{times}           % betûtípus
\usepackage{lmodern}          %   vagy inkább ez

\usepackage[margin=2.5cm,left=3.5cm,includeheadfoot]{geometry}
                              % margók
\usepackage{graphicx}         % képekhez
\usepackage{setspace}         % sorköz
\onehalfspacing               % másfeles




\begin{document}

% ------------------------------------------------------------------------------
% Címlap

\begin{titlepage}

\noindent
\parbox[m]{0.2\textwidth}{
%\includegraphics[width=0.2\textwidth]{elte_cimer_ff.eps}     % fekete-feh�r
 \includegraphics[width=0.2\textwidth]{elte_cimer_szines.eps} % sz�nes
}
\hfill
\parbox[m]{0.7\textwidth}{
\begin{center}
\begin{large}
\textsc{
E�tv�s Lor�nd Tudom�nyegyetem\\
\vspace{0.5pc}
Informatikai Kar\\
\vspace{0.5pc}
Numerikus Anal�zis Tansz�k\\
}
\end{large}
\end{center}
}

\vspace{1pc}
\hrule

\vfill

\begin{center}
{\LARGE A Szakdolgozat c�me}
\end{center}

\vfill

\noindent
\hspace*{0.05\textwidth}
\parbox{0.45\textwidth}{
{\it T�mavezet�:}
\bigskip

{\Large L�csi Levente}
\smallskip

tan�rseg�d
}
\hfill
\parbox{0.45\textwidth}{
{\it K�sz�tette:}
\bigskip

{\Large Minta Alad�r}
\smallskip

programtervez� informatikus BSc
}


\vfill

\begin{center}
{\large {\it Budapest, 2012}}
\end{center}

\end{titlepage}


% ------------------------------------------------------------------------------
% Témabejelentõ

\vspace*{\fill}
\begin{center}
Temabejelento
\end{center}
\vfill
\thispagestyle{empty}
\newpage
\setcounter{page}{1}

% ------------------------------------------------------------------------------
% Tartalomjegyzék

\tableofcontents

% ------------------------------------------------------------------------------


\chapter{Introduction}

A Bevezeto a temavalasztas indoklasat és a megoldando feladat rovid, kozertheto leirasat tartalmazza.


% ------------------------------------------------------------------------------


\chapter{Felhasználói dokumentáció}

A Felhasználói dokumentáció tartalmazza
\begin{itemize}
\item a megoldott probléma rövid megfogalmazását,
\item a felhasznált módszerek rövid leírását,
\item a program használatához szükséges összes információt.
\end{itemize}


% ------------------------------------------------------------------------------


\chapter{Fejlesztõi dokumentáció}

A Fejlesztõi dokumentáció tartlmazza
\begin{itemize}
\item a probléma részletes specifikációját,
\item a felhasznált módszerek részletes leírását, a használt fogalmak definícióját,
\item a program logikai és fizikai szerkezetének leírásár (adatszerkezetek, adatbázisok, modulfelbontás),
\item a tesztelési tervet és a tesztelés eredményeit.
\end{itemize}


% ------------------------------------------------------------------------------


% ------------------------------------------------------------------------------

\begin{thebibliography}{9}
\bibitem{v} Valaki, valami.
\end{thebibliography}

\end{document}
