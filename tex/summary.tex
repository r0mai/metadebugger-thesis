
\chapter{Summary} \label{summary}

\section{Results}

We implemented a tool which can aid C++ template metaprogram developers in
their work. Metadebugger can be used to debug metaprograms interactively and in
a flexible manner. Its features include stepping forward and backwards through
the individual template instantiations, printing back- and forwardtraces,
and placing and continuing until breakpoints.

Metashell version 2.0.0 was released in November, 2014\cite{github-releases}
and is the first release to include Metadebugger.

To this date two talks were given which included Metadebugger, both by Ábel
Sinkovics and me. A talk titled "Template Metaprogramming With Better Tools"
was given on December 1st, 2014 on a C++ Community Meetup in
Budapest\cite{cpp-meetup} and another was presented on December 6th, 2014 on
the MeetingC++ conference in Berlin with the title "Interactive Metaprogramming
Shell Based on Clang"\cite{meeting-cpp}. The second half of the talks consisted
of a brief introduction to Metadebugger and its internal structure, and a demo
showcasing the most important features.

\section{Possible further improvements}

\begin{itemize}
    \item
        Ability to debug metaprograms, even if the compilation failed, but
        templight produced output.
    \item
        Make the source location of point of instantiation available to the
        user.
    \item
        Smart formatting and indentation of very long template type names.
\end{itemize}
