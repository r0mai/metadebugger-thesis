
\chapter{Introduction} \label{introduction}

The tool which resulted from this thesis is called Metadebugger. This name will
be used in this document. What exactly Metadebugger is will be defined later.

\section{Motivation}

Programs often don't work the way their programmers intended. While some
problems can be fixed just by observing the produced output, most of them
require tools that enable the programmer to inspect what happens exactly
during runtime. These tools are called debuggers.

Most popular programming languages and runtime environments provide debuggers
for the programmers. Unfortunately no such tool existed for C++ Template
Metaprograms\cite{cpp14}. Metaprogram developers used to rely on the cryptic
and sometimes innumerable pages long error and warning messages produced by the
compiler to debug their code. Decoding these error messages can take hours or
sometimes even days for template heavy code\cite{sinkovics-phd,boost-spirit}.
Clearly, there is a need for a tool which makes this process easier.

Metadebugger tries to fill this gap by providing an easy to use tool, which can
be used to inspect what happens while a C++ compiler compiles a metaprogram.

Metadebugger uses Clang with Templight behind the scenes to gather the
necessary information about the evaluated metaprograms.

A talk titled "Template Metaprogramming With Better Tools" was given by Ábel
Sinkovics and me on December 1st, 2014 on a C++ Community
Meetup in Budapest\cite{cpp-meetup}. On December 6th, 2014 a similar talk
titled "Interactive Metaprogramming Shell Based on Clang" by the same speakers
was given in Berlin, on the Meeting C++ 2014 conference\cite{meeting-cpp}.
The second half of the talks consisted of a brief introduction to Metadebugger
and its internal structure, and a demo showcasing the most important features.

\section{Thesis structure}

The Introduction chapter \ref{introduction} gives a short motivation for
Metadebugger.

In the User documentation chapter \ref{userdoc}, I will describe the
installation procedure, then I will present the main concepts required to
understand how to use Metadebugger along with a guide which will show the user
the basic usage through examples. This chapter also contains the command
reference which is built into Metadebugger.

In the Developer documentation chapter \ref{devdoc} detailed description of the
main algorithms and data structures used are presented. I will also list the
used external libraries and what they were used for.

The Testing chapter \ref{testing} lists with short descriptions all the tests
used and created while developing Metadebugger.

In the Related work chapter \ref{relatedwork} I will list some work done by
other people related to my work.

In the last, Summary chapter \ref{summary} I will summarize the results
achieved, and what possible further improvements are planned for the future.

Throughout this document, the following metaprogram is used as an example or to
present various functionalities:

\begin{figure}[H]
    \includecode{src/fibonacci.hpp}
    \caption{Fibonacci metaprogram}
\end{figure}

This C++ template metaprogram can calculate the n-th Fibonacci number
(\(F_n\)), which is defined by the following recurrence relation:

\begin{figure}[H]
    \[
        F_n = F_{n-1} + F_{n-2} \quad \forall n \in \mathbb{N}: n \geq 2 \\
    \]
    \[
        F_0 = 0
    \]
    \[
        F_1 = 1
    \]
    \caption{Definition of the Fibonacci numbers}
\end{figure}
