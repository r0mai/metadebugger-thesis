
\chapter{Introduction}

The tool which resulted from this thesis is called Metadebugger. This name will
be used in this document. What exactly Metadebugger is will be later defined.

\section{Motivation}

Programs often don't work the way their programmers intended. While some
problems can be fixed just by observing the produced output, most of them
require tools that enable the programmer to inspect what happens exactly
during runtime. These tools are called debuggers.

Most popular programming languages and runtime environments provide debuggers
for the programmers. Unfortunately no such tool existed for C++ Template
Metaprograms\cite{cpp14}. Metaprogram developers used to rely on the cryptic
and sometimes innumerable pages long error and warning messages produced by the
compiler to debug their code. Decoding these error messages can take hours or
sometimes even days for template heavy code\cite{sinkovics-phd,boost-spirit}.
Clearly, there is a need for a tool which makes this process easier.

Metadebugger tries to fill this gap by providing an easy to use tool, which can
be used to inspect what happens while a C++ compiler compiles a metaprogram.

Metadebugger uses Clang with Templight behind the scenes to gather the
necessary information about the evaluated metaprograms.

\section{Thesis structure}

The introduction chapter (1) gives a short motivation for Metadebugger.

In the user documentation chapter (2), I will present the main concepts
required to understand how to use Metadebugger, then I will describe detailed
installation instruction and basic usage descriptions followed by the command
reference.

In the developer documentation chapter (3) detailed description of the main
algorithms and data structures used are presented. I will also list the used
external libraries and what they were used for.

Testing chapter (4) lists with short descriptions all the tests used and
created while developing Metadebugger.

In the summary chapter (5) I will summarize the results achieved and what
possible further improvements are planned for the future.

In the last chapter (6) I will list some work done by other people related to
my work.

Throughout this document, the following metaprogram is used as an example or to
present various functionalities:

\begin{figure}[H]
    \includecode{src/fibonacci.hpp}
    \caption{Fibonacci metaprogram}
\end{figure}

This C++ template metaprogram can calculate the n-th Fibonacci number
(\(F_n\)), which is defined by the following recurrence relation:

\begin{figure}[H]
    \[
        F_n = F_{n-1} + F_{n-2} \quad \forall n \in \mathbb{N}: n \geq 2 \\
    \]
    \[
        F_0 = 0
    \]
    \[
        F_1 = 1
    \]
    \caption{Definition of the Fibonacci numbers}
\end{figure}
