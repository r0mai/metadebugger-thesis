
% Maybe some stuff about template metaprogramming in this chapter?

\chapter{User documentation}

\section{Target audience}

This document expects the reader to have a basic understanding of template
metaprogramming in C++.

\section{Main concepts}

\section{Installation}

\section{Basic Usage}

\section{Command reference}

In the following section, the following notations are used: command parameters
that are in square brackets are optional. Parameters that are between angle
brackets have to be replaced by the user with something.

\subsection*{Evaluate}

Usage: \lstinline$evaluate [<type>]$

Evaluate and start debugging a new metaprogram.

If called with no arguments, then the last evaluated metaprogram will
be reevaluated.

Previous breakpoints are cleared.

Unlike metashell, evaluate doesn't use metashell::format to avoid
cluttering the debugged metaprogram with unrelated code. If you need
formatting, you can explicitly enter
\lstinline[language=C++]|metashell::format< <type> >::type| for the same
effect.

\subsection*{Step}

Usage: \lstinline$step [over] [n]$

Step the program.

Argument n means step n times. n defaults to 1 if not specified. Negative n
means step the program backwards.

Use of the \lstinline$over$ qualifier will jump over sub instantiations.

\subsection*{Rbreak}

Usage: \lstinline$rbreak <regex>$

Add breakpoint for all types matching \lstinline$<regex>$.

\subsection*{Continue}

Usage: \lstinline$continue [n]$

Continue program being debugged.

The program is continued until the nth breakpoint or the end of the program is
reached. n defaults to 1 if not specified. Negative n means continue the
program backwards.

\subsection*{Forwardtrace}

Usage: \lstinline$forwardtrace|ft [full] [n]$

Print forwardtrace from the current point.

Use of the full qualifier will expand Memoizations even if that instantiation
path has been visited before.

The n specifier limits the depth of the trace. If n is not specified, then the
trace depth is unlimited.

\subsection*{Backtrace}

Usage: \lstinline$backtrace|bt$

Print backtrace from the current point.

\subsection*{Help}

Usage: \lstinline$help [<command>]$

Show help for commands.
If \lstinline$<command>$ is specified, show a list of all available commands.

\subsection*{Quit}

Usage: \lstinline$quit$

Quit metadebugger.

