
% Maybe some stuff about template metaprogramming in this chapter?

\chapter{User documentation}

\section{Target audience}

This document expects the reader to have a basic understanding of template
metaprogramming in C++.

\section{Main concepts}

\section{Installation}

Metashell supports all major operating systems. In this section, installation
instructions only for Linux is described.

\subsection{Dependencies}

Install the dependent libraries and tools:

\begin{itemize}
    \item Readline
    \item Termcap
    \item CMake
\end{itemize}

Build Clang with Templight\cite{templight}:

\begin{itemize}
    \item \lstinline$mkdir build$
    \item \lstinline$cd build$
    \item \lstinline$cmake ../llvm -DLIBCLANG_BUILD_STATIC=ON$
    \item \lstinline$make clang$
    \item \lstinline$make libclang$
    \item \lstinline$make libclang_static$
\end{itemize}

\subsection{Building}

Now compile Metashell. In the source directory run the following commands:

\begin{itemize}
    \item \lstinline$mkdir bin$
    \item \lstinline$cd bin$
    \item \lstinline$cmake ..$
    \item \lstinline$make$
\end{itemize}

\subsection{Running tests}

To make sure everything will work correctly, running tests is advised:

\begin{itemize}
    \item \lstinline$test/metashell_test$
\end{itemize}

\section{Basic Usage}

%TODO how to start the program: ./app/metashell

\section{Command reference}

In the following section, the following notations are used: command parameters
that are in square brackets are optional. Parameters that are between angle
brackets have to be replaced by the user with something.

\subsection{evaluate}

Usage: \verb$evaluate [-full] [<type>]$

Evaluate and start debugging a new metaprogram.

Evaluating a metaprogram using the \verb$-full$ qualifier will expand all
Memoization events.

If called without <type>, then the last evaluated metaprogram will be
reevaluated.

Previous breakpoints are cleared.

Unlike metashell, evaluate doesn't use metashell::format to avoid cluttering
the debugged metaprogram with unrelated code. If you need formatting, you can
explicitly enter \verb$metashell::format< <type> >::type$ for the same effect.

\subsection{step}

Usage: \verb$step [over] [n]$

Step the program.

Argument n means step n times. n defaults to 1 if not specified.
Negative n means step the program backwards.

Use of the \verb$over$ qualifier will jump over sub instantiations.

\subsection{rbreak}

Usage: \verb$rbreak <regex>$

Add breakpoint for all types matching \verb$<regex>$.



\subsection{continue}

Usage: \verb$continue [n]$

Continue program being debugged.

The program is continued until the nth breakpoint or the end of the program
is reached. n defaults to 1 if not specified.
Negative n means continue the program backwards.

\subsection{forwardtrace}

Usage: \verb$forwardtrace|ft [n]$

Print forwardtrace from the current point.

The n specifier limits the depth of the trace. If n is not specified, then the
trace depth is unlimited.

\subsection{backtrace}

Usage: \verb$backtrace|bt $

Print backtrace from the current point.



\subsection{help}

Usage: \verb$help [<command>]$

Show help for commands.

If <command> is not specified, show a list of all available commands.

\subsection{quit}

Usage: \verb$quit $

Quit metadebugger.





