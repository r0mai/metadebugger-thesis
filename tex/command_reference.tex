\subsection{evaluate}

Usage: \verb$evaluate [-full] [<type>]$

Evaluate and start debugging a new metaprogram.

Evaluating a metaprogram using the \verb$-full$ qualifier will expand all
Memoization events.

If called without <type>, then the last evaluated metaprogram will be
reevaluated.

Previous breakpoints are cleared.

Unlike metashell, evaluate doesn't use metashell::format to avoid cluttering
the debugged metaprogram with unrelated code. If you need formatting, you can
explicitly enter \verb$metashell::format< <type> >::type$ for the same effect.

\subsection{step}

Usage: \verb$step [over] [n]$

Step the program.

Argument n means step n times. n defaults to 1 if not specified.
Negative n means step the program backwards.

Use of the \verb$over$ qualifier will jump over sub instantiations.

\subsection{rbreak}

Usage: \verb$rbreak <regex>$

Add breakpoint for all types matching \verb$<regex>$.



\subsection{continue}

Usage: \verb$continue [n]$

Continue program being debugged.

The program is continued until the nth breakpoint or the end of the program
is reached. n defaults to 1 if not specified.
Negative n means continue the program backwards.

\subsection{forwardtrace}

Usage: \verb$forwardtrace|ft [n]$

Print forwardtrace from the current point.

The n specifier limits the depth of the trace. If n is not specified, then the
trace depth is unlimited.

\subsection{backtrace}

Usage: \verb$backtrace|bt $

Print backtrace from the current point.



\subsection{help}

Usage: \verb$help [<command>]$

Show help for commands.

If <command> is not specified, show a list of all available commands.

\subsection{quit}

Usage: \verb$quit $

Quit metadebugger.



